\documentclass[11pt,a4paper]{article}
\usepackage[utf8]{inputenc}
\usepackage{amsmath}
\usepackage{amsfonts}
\usepackage{amssymb}
\usepackage{mathtools}
\usepackage{graphicx}
\usepackage{array}

\usepackage{gensymb}
\usepackage{biblatex} %Imports biblatex package
\addbibresource{sample.bib} %Import the bibliography file
\graphicspath{ {./images/} }
\newcommand{\norm}[1]{\left\lVert#1\right\rVert}
\title{Seismology - Summer Internship 2022 at GFZ Potsdam}
\author{Moshe Beutel}
\date{\today}

\begin{document}
\maketitle

\section{Overview}
Seismic earthquake research is full of annotated data. Open data sets from seismographic stations contains millions of manually annotated interesting events and is open to seismographic research community.
Efficient deep learning models developed in the last 4 years to deal with the following main tasks:\textit{ Earthquake detection, Phase Identification} and \textit{  Onset Time Picking} which correspond to the general algorithmic tasks of: Detection, Classification and estimation respectively.

\section{Tasks}
The tasks we are examining are the following:

\begin{center}

\begin{tabular}{ | m{4cm} | m{3cm}| m{3cm} |m{2cm} |} 
 \hline
 Task & Input & Output & Metric \\ 
 \hline\hline
 Event Detection & 30s window of seismic waveform & Contains First Arrival & AUC (or F1) \\ 
 \hline
 Phase Identification & 10s window of seismic waveform & Detrmine P or S & MCC \\
 \hline
 Onset Time Picking & 10s window contains one S or P wave (known) & Determine Onset Time & RMSE \\
 \hline
\end{tabular}
\end{center}

\section{Terminology}

\begin{itemize}
\item \textbf{P and S Waves} - Short for primary ad secondary waves. \textit{Body Waves} are energy travelling through solid volumes and \textit{Surface Waves} travel through free surfaces. The Body waves travel faster hence called primary or P Waves and the - slower - surface waves  are called Secondary or S Waves.

\item \textbf{Seismic Wave Equation} - 

%\item \textbf{Snell's Law} - A plane wave strikes a horizontal interface between 2 homogeneous layers of velocities $v_{1}, v_{2}$ respectively changes its angle at the interface $\theta_{1} \Rightarrow \theta_{2}$  to preserve the timing of the wavefronts across the interface.
%$$p = u_{1}sin\theta_{1} = u_{2}sin \theta_{2}$$
%where:
%\begin{itemize}
%\item $p$ is termed the \textit{ray parameter} (or horizontal slowness) and it remains unchanged.
%\item $u_{i} = \frac{1}{v_{i}}$ is termed the \textit{slowness} 
%\end{itemize}
%
%\item \textbf{Turning Point} - The point where the ray is no longer propagating down the layers ($\theta = 90\degree$).

\item \textbf{Arrival Time} - The time of first discernible motion of a seismic phase.

\item \textbf{Picking} - Measuring (Estimating ???) the arrival time


\end{itemize}


\section{Models}
The following models were tested in the benchmark:
\begin{itemize}
\item BasicPhaseAE (Woollam et al., 2019)
\item CNN-RNN Earthquake Detector (CRED; Mousavi, Zhu, et al., 2019)
\item DeepPhasePick(DPP; Soto \& Schurr, 2021)
\item Earthquake transformer (EQTransformer; Mousavi et al., 2020)
\item PhaseNet (Zhu \& Beroza, 2019)
\end{itemize}
\section{Limitation}

The noted models, although preformed well on the given tasks using the defined metrics, are still limited in the view of real life applications like early warning scenarios.

\begin{itemize}
\item \textbf{Datasets Limitations}  -
\begin{itemize}
\item Uncertainties and nonuniqueness of manual labels owing to limited resolution, presence of noise, \textbf{different levels of expertise}, cognitive biases, and inherent ambiguity of
tasks is a limiting factor. In image object classification for example tasks this is generally non-issue because normally most annotators would agree about pictures of cars,cats,tables and other daily life objects.
\item Not all seismic signals classes and typical noise are covered in the datasets - e.g. data from nodal seismometers at local distances, mine blasts, or volcanic signals.
\end{itemize}

 
\item \textbf{Tasks Limitation} - The tasks defined above does not exactly represent real life scenarios where:
\begin{itemize}
\item There are no defined time windows
\item More than one event may occur in a given time frame
\item The metrics defined does not take into account how early the tested algorithms would be able to identify an event onset
\item In continous time setup the false positive rate needs to be significantly lower than in post-processing.

\end{itemize}

\item Transfer Learning - It is yet unclear which datasets are most suitable for pretraining models

\end{itemize}


\section{Research Question Formulation}
The research question subjeect for this internship term deals with quantifying the uncertainty of a given model performence.
The 6-week time frame is ofcourse not suitable to solve that big problem so the exact task is yet to define





\end{document}